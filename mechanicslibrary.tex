\usetikzlibrary{shapes,arrows}
\tikzstyle{myarrow} = [-latex, thick, rounded corners=5.0]
\tikzstyle{myarrow_notrounded} = [-latex, thick]
\tikzstyle{myline} = [-, thick, rounded corners = 5.0]

% Define common color palette for mechanics problems.
\definecolor{darkred}   {rgb}{0.5, 0.1, 0.1}
\definecolor{medred}    {rgb}{0.8, 0.5, 0.5}
\definecolor{lightred}  {rgb}{1.0, 0.75, 0.75}

\definecolor{darkgreen} {rgb}{0.1, 0.5, 0.1}
\definecolor{medgreen}  {rgb}{0.5, 0.8, 0.5}
\definecolor{lightgreen}{rgb}{0.75, 1.0, 0.75}

\definecolor{darkblue}  {rgb}{0.1, 0.1, 0.5}
\definecolor{medblue}   {rgb}{0.5, 0.5, 0.8}
\definecolor{lightblue} {rgb}{0.75, 0.75, 1.0}


\newcommand{\pgfextractangle}[3]{
    % Obtains the angle between two coordinates, setting it to the first input.
    % 1: variable to set to the angle (e.g. \angle).
    % 2: point 1.
    % 3: point 2.

    \pgfmathanglebetweenpoints{\pgfpointanchor{#2}{center}}
                              {\pgfpointanchor{#3}{center}}
    \global\let#1\pgfmathresult  
}

\newcommand{\pgfextractdist}[3]{
    % Obtains the distance between two points, setting it to the first input.
    % 1: variable to set to the distance (e.g. \dist).
    % 2: point 1.
    % 3: point 2.
    \pgfpointdiff{\pgfpointanchor{#2}{center}}
                 {\pgfpointanchor{#3}{center}}
    \global\let#1\pgfmathresult  
}

\newcommand{\centerofmass}[1]{
    % Draws a center of mass circle.
    % 1: position.
    % TODO fix vertical 'overlap' of black between quadrants I and III.

    \draw[fill] (#1) -- +(0.1, 0) arc (0:90:0.1);
    \draw[fill] (#1) -- +(-0.1, 0) arc (180:270:0.1);
    \draw[fill=white] (#1) -- +(0,0.1) arc (90:180:0.1);
    \draw[fill=white] (#1) -- +(0,-0.1) arc (270:360:0.1);
    \draw (#1) circle (0.1);
}

\newcommand{\labeledArrow}[5]{
    % Draws an arrow with a label
    % 1: START point.
    % 2: END point.
    % 3: Label
    % 4: Label position offset from START point
    % 5: color
	\draw [->, very thick, draw=#5] (#1) -- (#2);
	\node (#3) at ($(#1)+(#4)$) []
    {\textcolor{#5}{#3}};
}

\newcommand{\bottomwall}[4]{
    % Draws a horizontal rigid wall intended to be on bottom of diagram.
    % 1: position of the top left corner of the wall.
    % 2: length of the wall.
    % 3: thickness of the wall.
    % 4: color of the wall.

    % Draw the hashed part.
    \draw[white,
        pattern color=#4,
        pattern=north east lines
        ]
        (#1) rectangle +(#2,-#3);

    % Draw the line of the wall.
    \draw[very thick, #4] (#1) -- +(right:#2);
}

\newcommand{\leftwall}[4]{
    % Draws a vertical rigid wall intended to be on the left side of diagram.
    % 1: position of the bottom right corner of the wall.
    % 2: length of the wall.
    % 3: thickness of the wall.
    % 4: color of the wall.

    % Draw the hashed part.
    \draw[white,
        pattern color=#4,
        pattern=north east lines
        ]
        (#1) rectangle ($(#1) + (-#3,#2)$);

    % Draw the line of the wall.
    \draw[very thick, #4] (#1) -- +(up:#2);
}

\newcommand{\bottomleftwall}[5]{
    % Draws a horizontal and vertical wall, intended to be on
    % the bottom left of the diagram.
    % 1: position of the bottom right corner of the wall, INSIDE the 'room'.
    % 2: width of the bottom wall.
    % 3: height of the left wall.
    % 4: thickness of the wall.
    % 5: color of the wall.

    % Draw the hashed part.
    \draw[white,
        pattern color=#5,
        pattern=north east lines
        ]
        (#1) -- ++(right:#2) -- ++(down:#4) -- ++(left:#2 + #4) -- 
            ++(up:#3 + #4) -- ++(right:#4) -- cycle;

    % - Draw the line of the wall.
    % Horizontal.
    \draw[very thick, #5] (#1) -- +(right:#2);
    % Vertical.
    \draw[very thick, #5] (#1) -- +(up:#3);
}

\newcommand{\cart}[6]{
    % Draws a cart, with a center of mass, and defines a point for the center
    % of mass of the cart, '<name>cm', that can be used elsewhere in the code.
    % 1: one-letter capital name (e.g. A)
    % 2: position of lower left corner of cart.
    % 3: width of rectangle
    % 4: height of rectangle
    % 5: wheel diameter
    % 6: color.

    % Draw the body, and both wheels.
    % TODO remove Apply fill and draw settings to the rectangle and both wheels.
    % Draw the border as darker than the fill.
    
    % - Draw all of the cart.
    \filldraw[fill=#6, very thick, draw=#6!40!black]
        % Body.
        (#2)++(0, #5) rectangle +(#3, #4)
        % Left wheel.
        (#2)++(0.5 * #5, 0.5 * #5) circle (0.5 * #5)
        %% Right wheel.
        (#2)++(#3 - 0.5 * #5, 0.5 * #5) circle (0.5 * #5);

    % Create a center of mass coordinate, and draw the COM symbol.
    \coordinate (#1cm) at ($(#2) + (0, #5) + (0.5 * #3, 0.5 * #4)$);
    \centerofmass{#1cm}
}
 
\newcommand{\rod}[6]{
    % Draws a rod, and defines a point for the center
    % of mass of the rod, '<name>cm', that can be used elsewhere in the code.
    % 1: one-letter capital name (e.g. A)
    % 2: position of one end of rod.
    % 3: length of the rod.
    % 4: thickness of the rod.
    % 5: angle of the rod from the horizontal.
    % 6: color.
    
    % Draw the rod.
    \filldraw[fill=#6, very thick, draw=#6!40!black, rotate=#5]
        ($(#2) + (0,0.5 * #4)$) rectangle +(#3, -#4);

    % Create a center of mass coordinate, and draw the COM symbol.
    \path (#2) -- coordinate (#1cm) +(#5:#3);
    \centerofmass{#1cm}
}

\newcommand{\hinge}[2]{
    % Draws a hinge.
    % 1: position of the hinge.
    % 2: diameter of the hinge to be drawn.

    \fill (#1) circle (#2);
}

\newcommand{\point}[3]{
    % Draws a point.
    % 1: position of the hinge.
    % 2: diameter of the hinge to be drawn.

    \fill[fill=#3, very thick, draw=#3!40!black] (#1) circle (#2);
}


\newcommand{\framelabel}[6]{
    % Creates a frame label: a letter in a circle with an arrow pointing
    % to the rigid body. In order to do its job, it defines the coordinate
    % '<name>node'.
    % 1: name.
    % 2: point to which the arrow points.
    % 3: point FROM which the arrow points.
    % 4: direction from which the arrow leaves the node (e.g. south).
    % 5: bend of the edge (e.g. bend right).
    % 6: color.

    \path (#3) node[shape=circle, minimum size=4mm, inner sep=0pt, draw=#6, fill=white] 
    (#1node) {\textcolor{#6}{#1}};
    \path[<-, #6, thick] (#2) edge [#5] (#1node.#4);
}

\newcommand{\pointlabel}[3]{
    % Creates a label with no arrow: a letter in a circle.
    % In order to do its job, it defines the coordinate
    % '<name>node'.
    % 1: name.
    % 2: location
    % 3: color.

    \node (#1) at (#2) [shape=circle, minimum size=4mm, inner sep=0pt, draw=#3, fill=white]
    {\textcolor{#3}{#1}};
}

\newcommand{\basisXY}[5]{
    % Draws a dextral orthogonal vector basis, with x and y in the plane.
    % 1: letter.
    % 2: center.
    % 3: rotation from the horizontal of the x component.
    % 4: color.
    \basisInPlaneTwo{#1}{#2}{#3}{#4}{#5}{x}{y}
}

\newcommand{\basisXYZ}[5]{
    % Draws a dextral orthogonal vector basis, with x and y in the plane and z out.
    % 1: letter.
    % 2: center.
    % 3: rotation from the horizontal of the x component.
    % 4: color.
    \basisInPlaneThree{#1}{#2}{#3}{#4}{#5}{x}{y}{z}
}

\newcommand{\basisInPlaneTwo}[7]{
    % Draws a dextral orthogonal vector basis (out of plane vector is omitted)
    % 1: letter.
    % 2: center.
    % 3: rotation from the horizontal of the x component.
    % 4: color.
    % 5: the label for the vector pointing "right".
    % 6: label for vector pointing "up".
    % 7: label for vector pointing "out".
    \draw[->, #4, very thick]
        (#2) -- +(#3:1*#5) node[above] {$\mathbf{\hat{#1}}_\mathrm{\textbf{#6}}$};
    \draw[->, #4, very thick]
        (#2) -- +(#3 + 90:1*#5) node[above] {$\mathbf{\hat{#1}}_\mathrm{\textbf{#7}}$};
}

\newcommand{\basisInPlaneThree}[8]{
    % Draws a dextral orthogonal vector basis with a vector out of page
    % 1: letter.
    % 2: center.
    % 3: rotation from the horizontal of the x component.
    % 4: color.
    % 5: the label for the vector pointing "right".
    % 6: label for vector pointing "up".
    % 7: label for vector pointing "out".
    \draw[->, #4, very thick]
        (#2) -- +(#3:1*#5) node[above] {$\mathbf{\hat{#1}}_\mathrm{\textbf{#6}}$};
    \draw[->, #4, very thick]
        (#2) -- +(#3 + 90:1*#5) node[above] {$\mathbf{\hat{#1}}_\mathrm{\textbf{#7}}$};
    \draw[->, #4, very thick] (#2)++(-60:0.2) arc (-60:150:0.2)
        node[below] {$\mathbf{\hat{#1}}_\mathrm{\textbf{#8}}$};
}

 
\newcommand{\dimsimple}[4]{
    % Draws a simple dimension, with the text in the middle of the dimension
    % line.
    % 1: text. 
    % 2: point 1.
    % 3: point 2.
    % 4: direction in which the text is placed around the midpoint of the dim.
    % TODO See if we can have the dimensions break the midway of the dimline, so
    % that we can save space, but this means doing something clever with TikZ.
    \draw[<->,>=latex,every rectangle node/.style={midway}]
        (#2) -- (#3) node[#4] {#1};
}
 
\newcommand{\dimoffset}[5]{
    % Draws an offset dimension, with the dimension bar offset a certain amount
    % and the text in the middle of the bar.
    % 1: text.
    % 2: point 1.
    % 3: point 2.
    % 4: offset, positive or negative.
    % 5: direction in which the text is placed around the midpoint of the dim.
    % TODO may only work for quadrants I and II.

    \pgfextractangle{\dimoffsetangle}{#2}{#3}
    \def\doanglesub{\dimoffsetangle - 90}
    \draw (#2) -- ++(\doanglesub:#4) coordinate (dimoset1) --+(\doanglesub:.2);
    \draw (#3) -- ++(\doanglesub:#4) coordinate (dimoset2) --+(\doanglesub:.2);
    \draw[<->,>=latex,every rectangle node/.style={midway}]
        (dimoset1) -- (dimoset2) node[#5] {#1};
}

\newcommand{\dimangle}[6]{
    % Angle dimension.
    % 1: text.
    % 2: center of rotation point.
    % 3: a point along one of the rays making the angle.
    % 4: distance from center of rotation to show dimension arc.
    % 5: angle about center away from the point.
    % 6: direction in which the text is placed around the midpoint of the dim.

    \pgfextractangle{\dimangleone}{#2}{#3}
    \draw[<->,>=latex,every rectangle node/.style={midway}]
        (#2)++(\dimangleone:#4) arc (\dimangleone:\dimangleone+#5:#4);
    \path (#2)++(\dimangleone:#4) arc (\dimangleone:\dimangleone+0.5*(#5):#4)
        node[#6] {#1};
}

