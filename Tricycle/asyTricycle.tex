\documentclass{standalone}
\usepackage{asymptote}

\begin{document}
\begin{asy}[width=10cm,height=10cm]

// See http://asymptote.sourceforge.net/doc/Import.html
import three;
import settings;

// Include has the effect of doing a drop-in replacement of the file
// For some reason I can't get import to work unless it is in the same directory.
include "../MechanicsLibrary.asy"; 
include "../SceneGraph.asy"; 


currentprojection=perspective( (15, 10,5 ), up=Z );
// showtarget=true, autoadjust=true, center=true );

// Construct the scene...
Node scene = Node( "root", null );

// N frame
Node frameN = Node( "N", scene );
scene.addChild(frameN);
frameN.addChild(createBasis("basisN", 1));

Node frameA = Node( "A", frameN );
frameN.addChild(frameA);
frameA.addChild(createBasis("basisA", 1));
frameA.rotation = rotate(-30, Z);
frameA.translation = shift(1,2,0);

// Draw the scene
scene.render(scale3(1));



// Draw the N basis.
//drawBasis(scale3(1), 1, "N");

/*
// ***************************************************************************
// I think we might be able to do MG-style frame definitions...
// Random experiment, just to see if it works:
real theta = -30; // argh, rotate takes a number of degrees, not radians
triple ax = rotate(theta, Y)*X;
triple ay = rotate(theta, Y)*Y;
triple az = rotate(theta, Y)*Z;
real len = 3;
triple N2 = (10, 0, 0);
draw(shift(N2)*(zero--(len*ax)),red+medium,Arrow3); 
draw(shift(N2)*(zero--(len*ay)),green+medium,Arrow3);
draw(shift(N2)*(zero--(len*az)),blue+medium,Arrow3);
draw(shift(N2)*(zero--len*(2*ax+az)), red, Arrow3);
// Cool it works! 
// TODO package this into useful library stuff and remove form this example.
// ***************************************************************************
*/

/*
// Define our position vectors... in the N frame.
triple zero = (0,0,0);
triple No = (0, 0, 0);
triple A = (0, 8, 20);
triple B = (0, 8, 0);
triple C = (15, 8, 0);
triple Qz = (0,0,8);
triple Qxz = (7,0,8);
triple Q = (7, 5, 8);

// ***************************************************************************
// I think we might be able to do MG-style frame definitions...
// Random experiment, just to see if it works:
real theta = -30; // argh, rotate takes a number of degrees, not radians
triple ax = rotate(theta, Y)*X;
triple ay = rotate(theta, Y)*Y;
triple az = rotate(theta, Y)*Z;
real len = 3;
triple N2 = (10, 0, 0);
draw(shift(N2)*(zero--(len*ax)),red+medium,Arrow3); 
draw(shift(N2)*(zero--(len*ay)),green+medium,Arrow3);
draw(shift(N2)*(zero--(len*az)),blue+medium,Arrow3);
draw(shift(N2)*(zero--len*(2*ax+az)), red, Arrow3);
// Cool it works! 
// TODO package this into useful library stuff and remove form this example.
// ***************************************************************************

// We can predefine surfaces as geometry plus affine transforms.
surface point = scale3(0.25)*unitsphere; // unitsphere has radius 1?
surface microphone = createCenteredBox((0.4, 0.8, 0.4));

// Draw the Walls
drawBox(No, A, identity4, mLightGreen);
drawBox(No, C, identity4, mLightBlue);

// Draw the points
draw(shift(No)*point,mBlack);
draw(shift(A)*point,mBlack);
draw(shift(B)*point,mBlack);
draw(shift(C)*point,mBlack);
draw(shift(Q)*point,mRed);
draw(shift(Qz)*point,mBlack);
draw(shift(Qxz)*point,mBlack);

// Draw the microphone
draw(shift(Q-Y*2/3)*microphone,mRed);

// Draw cables
draw((A--Q), black);
draw((B--Q), black);
draw((C--Q), black);

// Draw dimensions
draw(shift(Y/2)*(A--B), black, Arrows3, Bars3(Y), PenMargins2); // Arrows3 is bi-directional
//draw("$12$", shift(up)*(A--B), Z, Arrows3, Bars3(Y), PenMargins2); // I can't figue out labels...
draw(shift(Y/2)*(B--C), black, Arrows3, Bars3(Y));
draw(shift(X/2)*(C--(15,0,0)), black, Arrows3, Bars3(X));

draw(shift(Y/2)*(No -- (0,0,8)), black, Arrows3, Bars3(Y));
draw(shift(-Y/2)*((0,0,8) -- (7,0,8)), black, Arrows3, Bars3(Y));
draw(shift(X/2)*((7,0,8) -- (7,5,8)), black, Arrows3, Bars3(X));
*/




\end{asy}


\end{document}